% $Header: /Users/joseph/Documents/LaTeX/beamer/solutions/conference-talks/conference-ornate-20min.en.tex,v 90e850259b8b 2007/01/28 20:48:30 tantau $

\documentclass[16pt]{beamer}

% Copyright 2004 by Till Tantau <tantau@users.sourceforge.net>.
%
% In principle, this file can be redistributed and/or modified under
% the terms of the GNU Public License, version 2.
%
% However, this file is supposed to be a template to be modified
% for your own needs. For this reason, if you use this file as a
% template and not specifically distribute it as part of a another
% package/program, I grant the extra permission to freely copy and
% modify this file as you see fit and even to delete this copyright
% notice. 


\mode<presentation>
{
  %\usetheme{Warsaw}
  \usetheme{default}
  \setbeamercovered{invisible}
  %\setbeamercovered{transparent}
  %\usecolortheme{beaver}
  \beamertemplatenavigationsymbolsempty
  \usefonttheme{professionalfonts}
}


\usepackage[english]{babel}
\usepackage[latin1]{inputenc}
\usepackage{times}
\usepackage{listings}
\usepackage[T1]{fontenc}
% Or whatever. Note that the encoding and the font should match. If T1
% does not look nice, try deleting the line with the fontenc.


\lstdefinelanguage{ansibleyaml}%
  {morekeywords={name,hosts,tasks,command,handlers,template,with_items,%
      authorized\_key, copy, lineinfile, service, notify},%
   morekeywords=[2]{user,key,lookup,src,dest,owner,group,mode,%
      state, line, ansible_ssh_host, ansible_ssh_pass, ansible_ssh_user},
sensitive=false,
}


\lstset{%
  %basicstyle=\footnotesize,
  breakatwhitespace=true,
  breaklines=true,
  keywordstyle=\bfseries\color{green!40!black},
  keywordstyle={[2]\color{blue}},
  frame=single,
  language=ansibleyaml} 

\lstdefinestyle{custom_hosts}{%
  frame=single,
  language=ansibleyaml,
  basicstyle=\footnotesize\ttfamily
  }

\title % (optional, use only with long paper titles)
{Test environment configuration with Ansible.}

\author[Vikentsi Lapa] % (optional, use only with lots of authors)
{Vikentsi Lapa}
% - Give the names in the same order as the appear in the paper.
% - Use the \inst{?} command only if the authors have different
%   affiliation.

\institute[EPAM Systems] % (optional, but mostly needed)
% {
% - Use the \inst command only if there are several affiliations.
% - Keep it simple, no one is interested in your street address.

\date[LVEE 2014] % (optional, should be abbreviation of conference name)
{ Linux Vacation / Eastern Europe, 2014}
% - Either use conference name or its abbreviation.
% - Not really informative to the audience, more for people (including
%   yourself) who are reading the slides online

\subject{Test environment configuration with Ansible.}
% This is only inserted into the PDF information catalog. Can be left
% out. 


% If you have a file called "university-logo-filename.xxx", where xxx
% is a graphic format that can be processed by latex or pdflatex,
% resp., then you can add a logo as follows:

% \pgfdeclareimage[height=0.5cm]{university-logo}{university-logo-filename}
% \logo{\pgfuseimage{university-logo}}


% If you wish to uncover everything in a step-wise fashion, uncomment
% the following command: 

%\beamerdefaultoverlayspecification{<+->}


\begin{document}

\begin{frame}
  \titlepage
\end{frame}

\begin{frame}{Outline}
  \tableofcontents
  % You might wish to add the option [pausesections]
\end{frame}


% Structuring a talk is a difficult task and the following structure
% may not be suitable. Here are some rules that apply for this
% solution: 

% - Exactly two or three sections (other than the summary).
% - At *most* three subsections per section.
% - Talk about 30s to 2min per frame. So there should be between about
%   15 and 30 frames, all told.

% - A conference audience is likely to know very little of what you
%   are going to talk about. So *simplify*!
% - In a 20min talk, getting the main ideas across is hard
%   enough. Leave out details, even if it means being less precise than
%   you think necessary.
% - If you omit details that are vital to the proof/implementation,
%   just say so once. Everybody will be happy with that.

\section{Introduction}

\subsection{Do you need to listen this presentation? }

\begin{frame}{How often do you configure system?}

  \begin{itemize}
  \item<+->
     User 
  \item<+->
     Administrator
  \item<+->
     Developer
  \item<+->
     Project manager
  \item<+->
     Tester
  \end{itemize}
\end{frame}

\subsection{Test environment description.}

\begin{frame}{Cluster File Systems Testing.}
  \begin{itemize}
  \item
  6 hosts per environment. 
  \item
  5 environments. Can be more.
  \end{itemize}
  How often do we reinstall?
  \begin{itemize}
  \item
  New builds or releases
  \item
  Multiple OS and different hardware (IB, 40GigE)
  \item
  Defects verification often requires new installation.
  \end{itemize}
\end{frame}

\begin{frame}{What to do?}
Automate with language.
  \begin{itemize}
  \item
  Python, Bash, Perl
  \item
  Insert you favourite here.
  \end{itemize}
or Special tools
  \begin{itemize}
  \item
    Chef
  \item
    Puppet
  \item
    SaltStake
  \item
    Ansible
  \end{itemize}
\end{frame}

\section{About Ansible}
\begin{frame}{What do you need to install?}
  \begin{itemize}
  \item
  Push model.
  \item
  On clients. Only SSH and Python
  \item
  On controller. Python 2.6+, paramiko, PyYAML, jinja2, httplib2
  \end{itemize}
\end{frame}

\subsection{Inventory file.}

\lstset{style=custom_hosts}
\lstset{basicstyle=\scriptsize\ttfamily}
\begin{frame}{Inventory.}
  \begin{itemize}
  \item
  INI file with hosts description
  \item
  only host, group, combine group, filters 
  \end{itemize}
  \lstinputlisting{listings/hosts}
\end{frame}

\subsection{Tasks file.}
\begin{frame}{Playbooks.}
  \begin{itemize}
  \item
  YAML file discribes actions on hosts
  \end{itemize}
  \lstinputlisting[firstline=1, lastline=9]{listings/module_service.yaml}
\end{frame}


\subsection{Modules.}

\begin{frame}[fragile]
\frametitle{Module authorized\_key}
  \begin{itemize}
  \item
  Adds or removes an SSH authorized key
  \end{itemize}
  \lstinputlisting{listings/module_authorized_key.yaml}
\end{frame}

\begin{frame}[fragile]
\frametitle{Module copy}
  \begin{itemize}
  \item
  Copies files to remote locations
  \end{itemize}
  \lstinputlisting{listings/module_copy.yaml}
\end{frame}


\begin{frame}[fragile]
\frametitle{Module service}
  \begin{itemize}
  \item
  Manage services.
  \end{itemize}
  \lstinputlisting[firstline=8, lastline=9]{listings/module_service.yaml}
  \lstinputlisting[firstline=14, lastline=15]{listings/module_service.yaml}
\end{frame}

\begin{frame}[fragile]
\frametitle{Module lineinfile}
  \begin{itemize}
  \item
  Ensure a particular line is in a file, or replace an existing line using a
  back-referenced regular expression.
  \end{itemize}
  \lstinputlisting{listings/module_lineinfile_example.yaml}
\end{frame}

\section{If you need additional functions}
\subsection {Additional language constructions.}
\begin{frame}{Multipe ways to add new functionality.}
  \begin{itemize}
  \item
    Scripts.
  \item
    Templates engine.
  \item
    Module in any language (return JSON), Python API
  \end{itemize}
\end{frame}

\subsection {Three ways of network configuration.}

\begin{frame}{Scripts.}
  \lstinputlisting{listings/assign_IP_10K_with_script.yaml}
\end{frame}

\begin{frame}{Loops and variables.}
  \lstinputlisting[firstline=2, lastline=8]{listings/10K-1}
  \lstinputlisting[firstline=2, lastline=5]{listings/setup_lan.yaml}
\end{frame}

\begin{frame}{Templates engine.}
  \lstinputlisting[firstline=2, lastline=8]{listings/10K-1}
  \lstinputlisting[firstline=1, lastline=3]{listings/ifcfg.j2}
  \lstinputlisting[firstline=7, lastline=9]{listings/setup_lan.yaml}
\end{frame}

\begin{frame}{Screen shot with cow.}
\end{frame}



\section*{Summary}

\begin{frame}{Summary}

  % Keep the summary *very short*.
  \begin{itemize}
  \item
    Easy to install, use, extend.
  \item
    More time for other activities.
  \item
    I hope you will try Ansible.
  \end{itemize}
  
  % The following outlook is optional.
  \vskip0pt plus.5fill
  \begin{itemize}
  \item
    Outlook
    \begin{itemize}
    \item
      Conditional delays.
    \item
      Testing with Ansible.
    \end{itemize}
  \end{itemize}
\end{frame}



% All of the following is optional and typically not needed. 
\appendix
\section<presentation>*{\appendixname}
\subsection<presentation>*{For Further Reading}

\begin{frame}[allowframebreaks]
  \frametitle<presentation>{For Further Reading}
    
  \begin{thebibliography}{10}
    
  \beamertemplatebookbibitems
  % Start with overview books.

  \bibitem{Author1990}
    A.~Author.
    \newblock {\em Handbook of Everything}.
    \newblock Some Press, 1990.
 
    
  \beamertemplatearticlebibitems
  % Followed by interesting articles. Keep the list short. 

  \bibitem{Someone2000}
    S.~Someone.
    \newblock On this and that.
    \newblock {\em Journal of This and That}, 2(1):50--100,
    2000.
  \end{thebibliography}
\end{frame}

\end{document}


