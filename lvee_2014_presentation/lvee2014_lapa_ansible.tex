% $Header: /Users/joseph/Documents/LaTeX/beamer/solutions/conference-talks/conference-ornate-20min.en.tex,v 90e850259b8b 2007/01/28 20:48:30 tantau $

\documentclass[16pt]{beamer}

% Copyright 2004 by Till Tantau <tantau@users.sourceforge.net>.
%
% In principle, this file can be redistributed and/or modified under
% the terms of the GNU Public License, version 2.
%
% However, this file is supposed to be a template to be modified
% for your own needs. For this reason, if you use this file as a
% template and not specifically distribute it as part of a another
% package/program, I grant the extra permission to freely copy and
% modify this file as you see fit and even to delete this copyright
% notice. 


\mode<presentation>
{
  %\usetheme{Warsaw}
  \usetheme{default}
  \setbeamercovered{invisible}
  %\setbeamercovered{transparent}
  %\usecolortheme{beaver}
  \beamertemplatenavigationsymbolsempty
  \usefonttheme{professionalfonts}
}


\usepackage[english]{babel}
\usepackage[latin1]{inputenc}
\usepackage{times}
\usepackage{listings}
\usepackage[T1]{fontenc}
% Or whatever. Note that the encoding and the font should match. If T1
% does not look nice, try deleting the line with the fontenc.


\lstdefinelanguage{ansibleyaml}%
  {morekeywords={name,hosts,tasks,command,handlers,template,with_items,%
      authorized\_key, copy, lineinfile, service, notify},%
   morekeywords=[2]{user,key,lookup,src,dest,owner,group,mode,%
      state, line, ansible_ssh_host, ansible_ssh_pass, ansible_ssh_user},
sensitive=false,
}


\lstset{%
  basicstyle=\normalsize,
  breakatwhitespace=true,
  breaklines=true,
  keywordstyle=\bfseries\color{green!40!black},
  keywordstyle={[2]\color{blue}},
  frame=single,
  language=ansibleyaml} 

\lstdefinestyle{custom_hosts}{%
  frame=single,
  language=ansibleyaml,
  %basicstyle=\huge
  %basicstyle=\footnotesize\ttfamily
  }

\title % (optional, use only with long paper titles)
{Test environment configuration with Ansible.}

\author[Vikentsi Lapa] % (optional, use only with lots of authors)
{Vikentsi Lapa}
% - Give the names in the same order as the appear in the paper.
% - Use the \inst{?} command only if the authors have different
%   affiliation.

\institute[EPAM Systems] % (optional, but mostly needed)
% {
% - Use the \inst command only if there are several affiliations.
% - Keep it simple, no one is interested in your street address.

\date[LVEE 2014] % (optional, should be abbreviation of conference name)
{ Linux Vacation / Eastern Europe, 2014}
% - Either use conference name or its abbreviation.
% - Not really informative to the audience, more for people (including
%   yourself) who are reading the slides online

\subject{Test environment configuration with Ansible.}
% This is only inserted into the PDF information catalog. Can be left
% out. 


% If you have a file called "university-logo-filename.xxx", where xxx
% is a graphic format that can be processed by latex or pdflatex,
% resp., then you can add a logo as follows:

% \pgfdeclareimage[height=0.5cm]{university-logo}{university-logo-filename}
% \logo{\pgfuseimage{university-logo}}


% If you wish to uncover everything in a step-wise fashion, uncomment
% the following command: 

%\beamerdefaultoverlayspecification{<+->}


\begin{document}

\begin{frame}
  \titlepage
\end{frame}

\begin{frame}{Outline}
  \tableofcontents
  % You might wish to add the option [pausesections]
\end{frame}


% Structuring a talk is a difficult task and the following structure
% may not be suitable. Here are some rules that apply for this
% solution: 

% - Exactly two or three sections (other than the summary).
% - At *most* three subsections per section.
% - Talk about 30s to 2min per frame. So there should be between about
%   15 and 30 frames, all told.

% - A conference audience is likely to know very little of what you
%   are going to talk about. So *simplify*!
% - In a 20min talk, getting the main ideas across is hard
%   enough. Leave out details, even if it means being less precise than
%   you think necessary.
% - If you omit details that are vital to the proof/implementation,
%   just say so once. Everybody will be happy with that.

\section{Introduction}

\subsection{Do you need to listen this presentation? }

\begin{frame}{How often do you configure system?}

  \begin{itemize}
  \item<+->
     User 
  \item<+->
     Administrator
  \item<+->
     Developer
  \item<+->
     Project manager
  \item<+->
     Tester
  \end{itemize}
\end{frame}

\subsection{Test environment description.}

\begin{frame}{Cluster File Systems Testing.}
  \begin{itemize}
  \item
  6 hosts per environment. 
  \item
  5 environments. Can be more.
  \end{itemize}
  How often do we reinstall?
  \begin{itemize}
  \item
  New builds or releases
  \item
  Multiple OS and different hardware (IB, 40GigE)
  \item
  Defects verification often requires new installation.
  \end{itemize}
\end{frame}

\begin{frame}{What to do?}
\begin{columns}[T]
  \begin{column}{.6\textwidth}
        Automate with 
        \begin{itemize}
        \item
        Python, Bash, Perl
        \item
        Insert you favourite here.
        \end{itemize}

        \vspace*{1\baselineskip}
        or use special tools
        \begin{itemize}
        \item
            Chef
        \item
            Puppet
        \item
            SaltStack
        \item
            Ansible
        \end{itemize}
    \end{column}
  \begin{column}{.4\textwidth}
    \includegraphics[scale=0.2]{images/Chef_Vertical_Reg_Without.png}

    \includegraphics[scale=0.1]{images/logo_PL_vertical_RGB_lg.png}

    \includegraphics[scale=1.9]{images/logo_salt_stack.png}

    \includegraphics[scale=0.5]{images/logo_ansible.png}
    \end{column}
\end{columns}
\end{frame}

\section{About Ansible}
\begin{frame}{What do you need to install?}
  \begin{itemize}
  \item
  On controller.
    \begin{itemize}
        \item
        Python 2.6+
        \item
        Paramiko 
        \item
        PyYAML 
        \item
        Jinja2
    \end{itemize}
  \item
  On clients. 
    \begin{itemize}
        \item
        Only SSH and Python. (Usually it is already installed.)
    \end{itemize}
  \item
  Push model.
  \end{itemize}
  \note {%
  Ansible by default manages machines over the SSH protocol.
  You need install it only to one machine. 

  Once Ansible is installed, it will not add a database, and there will be no
  daemons to start or keep running. You only need to install it on one machine
  (which could easily be a laptop) I use desktop.
  When Ansible manages remote machines, it does not leave software installed or running on them, so there’s no real question about how to upgrade Ansible when moving to a new version.
  }
\end{frame}



\subsection{Inventory file.}

\lstset{style=custom_hosts}
%\lstset{basicstyle=\scriptsize\ttfamily}
%\lstset{basicstyle=\large}
\begin{frame}{Inventory.}
  \begin{itemize}
  \item
  INI file with hosts description
  \item
  only host, group, combine group, filters 
  \end{itemize}
  \lstinputlisting{listings/hosts}
\end{frame}

\subsection{Tasks file.}
\begin{frame}{Playbooks.}
  \begin{itemize}
  \item
  YAML file describes actions on hosts
  \end{itemize}
  \lstinputlisting[firstline=1, lastline=9]{listings/module_service.yaml}
  \$ ansible-playbook -i hosts service.yaml
\end{frame}


\subsection{Modules.}

\begin{frame}[fragile]
\frametitle{Module authorized\_key}
  \begin{itemize}
  \item
  Adds or removes an SSH authorized key
  \end{itemize}
  \lstinputlisting{listings/module_authorized_key.yaml}
\end{frame}

\begin{frame}[fragile]
\frametitle{Module copy}
  \begin{itemize}
  \item
  Copies files to remote locations
  \end{itemize}
  \lstinputlisting{listings/module_copy.yaml}
\end{frame}


\begin{frame}[fragile]
\frametitle{Module service}
  \begin{itemize}
  \item
  Manage services.
  \end{itemize}
  \lstinputlisting[firstline=8, lastline=9]{listings/module_service.yaml}
  \lstinputlisting[firstline=14, lastline=15]{listings/module_service.yaml}
\end{frame}

\begin{frame}[fragile]
\frametitle{Module lineinfile}
  \begin{itemize}
  \item
  Ensure a particular line is in a file, or replace an existing line using a
  back-referenced regular expression.
  \end{itemize}
  \lstinputlisting{listings/module_lineinfile_example.yaml}
\end{frame}

\section{If you need additional functions}
\subsection {Additional language constructions.}
\begin{frame}{Multiple ways to add new functionality.}
  \begin{itemize}
  \item
    Python API
  \item
    Developing Dynamic Inventory Sources
  \item
    Developing Modules
  \item
    Developing Plugins
  \end{itemize}
 Task: Network interface configuration
  \begin{itemize}
  \item
    Scripts.
  \item
    Templates engine.
  \item
    Module (not implemented) 
  \end{itemize}
\end{frame}

\subsection {Add new functions for network configuration.}

\begin{frame}{Module script.}
  script - Runs a local script on a remote node after transferring it.
  \lstinputlisting{listings/assign_IP_10K_with_script.yaml}
  It is usually preferable to write Ansible modules than pushing scripts.
\end{frame}

\begin{frame}{Loops and variables.}
  \lstinputlisting[firstline=2, lastline=8]{listings/10K-1}
  \lstinputlisting[firstline=2, lastline=5]{listings/setup_lan.yaml}
\end{frame}

\lstset{basicstyle=\small}
\begin{frame}{Module templates.}
  \lstinputlisting[firstline=1, lastline=4]{listings/ifcfg.j2}
  \lstinputlisting[firstline=7, lastline=9]{listings/setup_lan.yaml}
  \lstinputlisting[firstline=2, lastline=8]{listings/10K-1}
\end{frame}


\section*{Summary}

\begin{frame}{Summary}

  % Keep the summary *very short*.
  \begin{itemize}
  \item
    Easy to install, use, extend.
  \item
    More time for other activities.
  \item
    I hope you will try Ansible.
  \end{itemize}
  \lstinputlisting{listings/cowsay}
  
\end{frame}


% All of the following is optional and typically not needed. 
\appendix
\section<presentation>*{\appendixname}
\subsection<presentation>*{For Further Reading}

\begin{frame}
  \frametitle<presentation>{For Further Reading}

  \begin{itemize}
  \item
    \href{ http://www.ansible.com/}{Ansible project}

    \url {http://www.ansible.com/}
  \item
    \href{ http://docs.ansible.com/ }{Documentation}

    \url{http://docs.ansible.com/}
  \item
    \href{https://github.com/ansible/ansible}{Source code}

    \url{https://github.com/ansible/ansible}
  \item
    \href{ https://github.com/nixuser/codesamples/tree/master/lvee_2014_presentation}{Presentation with listings}

    \url{https://github.com/nixuser/codesamples/tree/master/lvee_2014_presentation}
  \end{itemize}
\end{frame}


\end{document}

