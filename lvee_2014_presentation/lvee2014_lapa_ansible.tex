% $Header: /Users/joseph/Documents/LaTeX/beamer/solutions/conference-talks/conference-ornate-20min.en.tex,v 90e850259b8b 2007/01/28 20:48:30 tantau $

\documentclass{beamer}

% This file is a solution template for:

% - Talk at a conference/colloquium.
% - Talk length is about 20min.
% - Style is ornate.



% Copyright 2004 by Till Tantau <tantau@users.sourceforge.net>.
%
% In principle, this file can be redistributed and/or modified under
% the terms of the GNU Public License, version 2.
%
% However, this file is supposed to be a template to be modified
% for your own needs. For this reason, if you use this file as a
% template and not specifically distribute it as part of a another
% package/program, I grant the extra permission to freely copy and
% modify this file as you see fit and even to delete this copyright
% notice. 


\mode<presentation>
{
  \usetheme{Warsaw}
  % or ...

  \setbeamercovered{transparent}
  % or whatever (possibly just delete it)
}


\usepackage[english]{babel}
% or whatever

\usepackage[latin1]{inputenc}
% or whatever

\usepackage{times}
\usepackage[T1]{fontenc}
% Or whatever. Note that the encoding and the font should match. If T1
% does not look nice, try deleting the line with the fontenc.

\usepackage{listings}


\title % (optional, use only with long paper titles)
{Test environment configuration with Ansible.}

\author[Lapa] % (optional, use only with lots of authors)
{V.~Lapa\inst{1}}
% - Give the names in the same order as the appear in the paper.
% - Use the \inst{?} command only if the authors have different
%   affiliation.

\institute[EPAM Systems] % (optional, but mostly needed)
% {
% - Use the \inst command only if there are several affiliations.
% - Keep it simple, no one is interested in your street address.

\date[LVEE 2014] % (optional, should be abbreviation of conference name)
{ Linux Vacation / Eastern Europe, 2014}
% - Either use conference name or its abbreviation.
% - Not really informative to the audience, more for people (including
%   yourself) who are reading the slides online

\subject{Test environment configuration with Ansible.}
% This is only inserted into the PDF information catalog. Can be left
% out. 


% If you have a file called "university-logo-filename.xxx", where xxx
% is a graphic format that can be processed by latex or pdflatex,
% resp., then you can add a logo as follows:

% \pgfdeclareimage[height=0.5cm]{university-logo}{university-logo-filename}
% \logo{\pgfuseimage{university-logo}}


% Delete this, if you do not want the table of contents to pop up at
% the beginning of each subsection:
%\AtBeginSubsection[]
%{
%  \begin{frame}<beamer>{Outline}
%    \tableofcontents[currentsection,currentsubsection]
%  \end{frame}
%}


% If you wish to uncover everything in a step-wise fashion, uncomment
% the following command: 

%\beamerdefaultoverlayspecification{<+->}


\begin{document}

\begin{frame}
  \titlepage
\end{frame}

\begin{frame}{Outline}
  \tableofcontents
  % You might wish to add the option [pausesections]
\end{frame}


% Structuring a talk is a difficult task and the following structure
% may not be suitable. Here are some rules that apply for this
% solution: 

% - Exactly two or three sections (other than the summary).
% - At *most* three subsections per section.
% - Talk about 30s to 2min per frame. So there should be between about
%   15 and 30 frames, all told.

% - A conference audience is likely to know very little of what you
%   are going to talk about. So *simplify*!
% - In a 20min talk, getting the main ideas across is hard
%   enough. Leave out details, even if it means being less precise than
%   you think necessary.
% - If you omit details that are vital to the proof/implementation,
%   just say so once. Everybody will be happy with that.

\section{Introduction}

\subsection{ Do you need to listen this presentation? }

\begin{frame}{How often do you configure system?}

  \begin{itemize}
  \item
     User \texttt{itemize} a lot.
  \item
     Administrator
  \item
     Developer
  \item
     Project manager
  \end{itemize}
\end{frame}

\begin{frame}{Special tools.}
  \begin{itemize}
  \item
  \item
  \item
  \item
  \end{itemize}
\end{frame}

\subsection{Test environment description.}
\begin{frame}{Cluster File Systems testing.}
  \begin{itemize}
  \item
  Number of hosts
  \item
  How often do we reinstall?
  \item
  \end{itemize}
\end{frame}


\section{About Ansible}
\begin{frame}{What is difference?}
  \begin{itemize}
  \item
  No agents
  \end{itemize}
\end{frame}

\subsection{Inventory file.}
\begin{frame}{Inevntory.}
  \begin{itemize}
  \item
  INI file with hosts description
  \item
  only host, group, combine group, filters 
  \end{itemize}
\end{frame}

\subsection{Tasks file.}
\begin{frame}{Playbooks.}
  \begin{itemize}
  \item
  YAML file discribes actions on hosts
  \end{itemize}
\end{frame}

\lstdefinelanguage{ansibleyaml}%
  {morekeywords={name,%
      authorized\_key, copy},%
   morekeywords=[2]{user,key,lookup, src, dest,owner,group,mode},
sensitive=false,
}


\lstset{%
  %basicstyle=\footnotesize,
  breakatwhitespace=true,
  breaklines=true,
  keywordstyle=\bfseries\color{green!40!black},
  keywordstyle={[2]\color{blue}},
  frame=single,
  language=ansibleyaml} 

\begin{frame}[fragile]
\frametitle{Module authorized\_key}
  \begin{itemize}
  \item
  Adds or removes an SSH authorized key
  \end{itemize}
  \lstinputlisting{listings/auth_key.yaml}
\end{frame}

\begin{frame}[fragile]
\frametitle{Module copy}
  \begin{itemize}
  \item
  Copies files to remote locations
  \end{itemize}
  \lstinputlisting{listings/copy_task.yaml}
\end{frame}



\subsection{Extension.}

\begin{frame}{Extentions.}
\end{frame}

\section{If you need additional functions?}
\subsection {More language constractions}
\subsection {Write module for you task}


\section*{Summary}

\begin{frame}{Summary}

  % Keep the summary *very short*.
  \begin{itemize}
  \item
    The \alert{first main message} of your talk in one or two lines.
  \item
    The \alert{second main message} of your talk in one or two lines.
  \item
    Perhaps a \alert{third message}, but not more than that.
  \end{itemize}
  
  % The following outlook is optional.
  \vskip0pt plus.5fill
  \begin{itemize}
  \item
    Outlook
    \begin{itemize}
    \item
      Something you haven't solved.
      Conditional Delays 
    \item
      Testing with Ansible
    \end{itemize}
  \end{itemize}
\end{frame}



% All of the following is optional and typically not needed. 
\appendix
\section<presentation>*{\appendixname}
\subsection<presentation>*{For Further Reading}

\begin{frame}[allowframebreaks]
  \frametitle<presentation>{For Further Reading}
    
  \begin{thebibliography}{10}
    
  \beamertemplatebookbibitems
  % Start with overview books.

  \bibitem{Author1990}
    A.~Author.
    \newblock {\em Handbook of Everything}.
    \newblock Some Press, 1990.
 
    
  \beamertemplatearticlebibitems
  % Followed by interesting articles. Keep the list short. 

  \bibitem{Someone2000}
    S.~Someone.
    \newblock On this and that.
    \newblock {\em Journal of This and That}, 2(1):50--100,
    2000.
  \end{thebibliography}
\end{frame}

\end{document}


